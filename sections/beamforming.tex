\chapter{Beamforming}
\label{chapter:beamforming}
\chaptermark{Relative velocity reconstruction}



% ================================================================================
% CHAPTER OVERVIEW
% ================================================================================

\section{Chapter Overview}

In an era when electric fields can be sampled billions of 
times per second, radio telescopes are becoming almost entirely digital. 
Though the cost of constructing large single-dish telescopes is not expected to
decrease substantially, the cost of building large computing clusters is, 
which makes it economically and strategically sensible to 
point one's telescope in software, as with digital beamforming.
Beamforming is particularly essential to CHIME. The pulsar back-end will rely on
brute-force beamforming in order to track ten sources at a time, 24-7.  
The FRB experiment will FFT-beamform to generate 1024 fan-beams, 
in order to search them in real time for radio transients. And the cosmology 
experiment has always left itself the option of beamforming, whose 
computing cost scales as $N\log N$, as 
an alternative to the full $N^2$ correlation. This chapter outlines the 
basic theory behind digital beamforming, and describes the commissioning 
of the first beamformer on CHIME Pathfinder. This includes the synthesis 
of several different software packages, the implementation of an early scheduler, 
and an automated point-source calibration daemon that removes drifting instrumental 
gains in real-time. We will also detail early pulsar 
work and the creation of an ongoing VLBI FRB search between 
the DRAO and ARO.

% ================================================================================
% INTRODUCTION
% ================================================================================

\section{Introduction}


  
% ================================================================================
% THEORY AND IMPLEMENTATION
% ================================================================================
  
\section{Theory and Implementation}
\label{sec:theory}

Beamforming is a signal processing technique that allows for 
spatial filtering, and has greatly benefited a diverse set of fields 
from radar and wireless communications to radio astronomy.
Historically, this was  % Insert history

By coherently combining the voltages of a multi-element array, 
sensitivity can be allocated to small regions of the sky and 
the array's effective forward gain can be increased. The signal 
from each antenna, $x_n$, is multiplied by a complex weight whose 
phases, $\phi_{n}$, are chosen to destructively interfere radio waves 
in all directions but the desired pointing. The signals 
from all antennas are then combined to give the formed-beam 
voltage stream, $X_{\rm BF}$.

\begin{equation}
\label{eq-bf_sum}
X_{\rm BF} = \sum_{{n}=1}^N a_n e^{i\phi_{n}} x_n
\end{equation}

\noindent Here $a_n$ are real numbers that can be used to as 
amplitude weightings for the antennas. If we define a more 
general complex weighting, $w_n \equiv a_n e^{i\phi_{n}}$, and 
switch to vector notation, Eq.~\ref{eq-bf_sum} becomes,

\begin{equation}
X_{\rm BF} = \mathbf{w} \, \mathbf{x}^{\rm T} .
\end{equation}

\noindent In general, $X_{\rm BF}$ and $\mathbf{x}^{\rm T}$ will be 
functions of time and frequency. This is also true for $\mathbf{w}$,
unless one needs a static, non-tracking beam -- which is the case for the 
CHIME Pathfinder's transient search. We can write this explicitly as follows. 


\begin{align}
     \mathbf{w}_{\rm t \nu} &= \left (a_1(\nu) e^{i \phi_1(\nu)}, \, 
     a_2(\nu) e^{i \phi_2(\nu)}, ... \,, \,a_N(\nu) e^{i \phi_N(\nu)} \right )\\
     \mathbf{x}_{\rm t \nu} &= \left ( x_1(\rm{t}, \nu), \, x_2(\rm{t}, \nu), 
     ..., \, x_N(\rm{t}, \nu) \right )
% \mathbf{x}_{\rm t \nu} = \left ( x_1(\rm{t} \nu), \, 
% x_2(\rm{t} \nu)\, , ... \,, \,x_N(\rm{t} \nu) \right )
\end{align}

\subsection{Geometric phase}

We now need to calculate $\phi_n$ across the array.
Ignoring instrumental phases for now, one can compute the geometric 
phases for an antenna by projecting its position vector, $\mathbf{d}_n$, 
onto the pointing vector, $\hat{\mathbf{k}}$. This gives,

\begin{equation}
\label{eqn-phi_n}
\phi_n = \frac{2\pi}{\lambda} \, \mathbf{d}_n \cdot  {\mathbf{\hat{k}}}
\end{equation}

\noindent where we have taken $\mathbf{d}_n$ to be the baseline vector between 
feed $n$ and an arbitrary reference point, and $\phi_n$ is the corresponding 
phase difference. A sketch for this is shown in Fig.~\ref{fig-bf_diagram} on page 
\pageref{eqn2}.


%trim={<left> <lower> <right> <upper>}
\begin{figure}[H]
\label{fig-bf_diagram}
\begin{center}
\includegraphics[trim={1.in, 1.in, 2.5in, 1.in}, width=1\textwidth]{./figures/beamforming/beamforming_diagram.jpeg} 
\vspace{0.0cm}
\caption[abc]{Diagrammatic example of a three-element beamformer. The 
wavefront from a far-field point source arrives at each antenna 
at different times, but the delay is calculable given an array 
configuration and a direction to the object. Complex weights can 
be applied to each antenna's voltage time-stream to account 
for the geometric delay, allowing for the signals to be summed coherently.}  
\vspace{-0.4cm}   
\end{center}
\end{figure}

To calculate the projection $\mathbf{d}_n \cdot  {\mathbf{\hat{k}}}$, we 
need to go from celestial coordinates, in this case equatorial, to geographic 
coordinates. This requires only a source location, an observer location, and an 
observing time. For the latter we use local 
sidereal time (LST), which is the $RA$ of the local meridian. This can be determined  
by an observer's longitude and a time, e.g. a Coordinated Universal Time (UTC). 
A source's hour angle is simply the difference between $LST$ and its $RA$,

\begin{equation}
HA = LST - RA.
\end{equation}

We use the standard interferometric $(u, v, w)$ coordinate system 
to describe our baseline vector, $\mathbf{d}_n$. This is a 
right-handed coordinate system where $u$ and $v$ are in the plane 
whose normal is the zenith, and $w$ measures the vertical direction \citep{1986isra.book.....T}.
They are defined in numbers of wavelengths, with
$u = d_{\rm ew} / \lambda$, $v = d_{\rm ns} / \lambda$, 
and $w = d_{\rm vert} / \lambda$. Eqn~\ref{eqn-phi_n} can we expanded 
as,

\begin{align}   
\phi_n &= 2\pi \, (u, v, w) \cdot \mathbf{\hat{k}}\\
&= 2 \pi \left ( 
u \, \mathit{\mathbf{\hat{u}}} \cdot \mathbf{\hat{k}} + 
v \, \mathbf{\hat{v}} \cdot \mathbf{\hat{k}} + 
w \, \mathbf{\hat{w}} \cdot \mathbf{\hat{k}} 
\right ),
\end{align}

\noindent where each projection component can be obtained 
using spherical trigonometry. Though we do not go through the
derivation here, it is given by the following product,

\begin{equation}
\mathbf{d}_n \cdot  {\mathbf{\hat{k}}} = \lambda \begin{pmatrix}
u, & v, & w
\end{pmatrix}  \cdot \begin{pmatrix} 
-\mathrm{cos}\delta \,\mathrm{sin}HA \\ 
\, \mathrm{cos}(lat) \, \mathrm{sin}\delta - \mathrm{sin}(lat) \, \mathrm{cos}\delta \, \mathrm{cos}HA \,\\
\, \mathrm{sin}(lat) \, \mathrm{sin}\delta + \mathrm{cos}(lat) \, \mathrm{cos}\delta \, \mathrm{cos} HA\,
\end{pmatrix} .
\end{equation}

These phases are not only essential to the beamforming



\begin{align}
\rm{sin}(alt) = \rm{sin}(\delta)\, \rm{sin}(lat) + cos(\delta)\, cos(lat)\, cos(\rm HA)\\
\rm{cos}(az) = \frac{\rm{sin}\delta - \rm{sin}(alt) \,
\rm{sin}(lat)}{\rm{cos}(alt)\, \rm{cos}(lat)}
\end{align}

\begin{table}[]
\centering
\label{tab-coord_var}
\begin{tabular}{ll}
\multicolumn{1}{c}{\textbf{Variable}} & \multicolumn{1}{c}{\textbf{Coordinate}} \\ \hline
$\delta$                              & Source declination                      \\
$RA$                                    & Source right ascention                  \\
$LST$                                   & Local sidereal time                     \\
$HA$                                    & Source hour angle                       \\
$alt$                                   & Source altitude                         \\
$az$                                    & Source azimuth                          \\
$lat$                                   & Telescope latitude                      \\
$lon$                                   & Telescope longitude                    
\end{tabular}
\end{table}

\section{Pathfinder beamformer}

\subsection{Instrumental phases}
In a real experiment, if the voltages from each antenna, $x_n$, are summed 
without any adjustment as written in Eq~\ref{eq-bf_sum}, one should only 
expect noise and not a coherent beam. This is because we have assumed 
the wavefront's differential time-of-arrival across at array 
is the same time delay seen by the correlator. In fact each 
signal is further delayed by multiple steps in the signal chain. 
Digital phases in the electronics can be added by the LNAs and FLAs, and
coaxial cables, whose lengths vary by up to a meter, can rotate 
the signal by multiple radians. Therefore in order to coherently sum 
across the array and beamform, the instrumental phases must be removed. 
If $e_n$ is the true electric field on the 
sky as seen by each feed, then the thing we measure is the on-sky signal
altered by an effective gain, $g_n$, and a noise term, $n_n$.

\begin{equation}
     x_n = g_n e_n + n_n
\end{equation}

\noindent We have lumped several terms into $g_n = |g_n| e^{i \phi_{g_n}}$, 
which is composed of a pointing-dependent beam term
and any complex gain introduced after light hits the cylinder. 
Since we really only care about the phase, we can decompose $\arg(g_n)$
as,

\begin{equation}
\phi_{g_n} = \phi_{\rm beam} + \phi_{\rm an} + \phi_{\rm e} + \phi_{\rm fpga} 
\end{equation}

\noindent where $\phi_{\rm beam}$ is the beam's phase for a given pointing, 
$\phi_{\rm an}$ comes from the analogue chain (dual-pol feed, coax, etc.),  
and $\phi_{\rm fpga}$ are phases applied in the $F$-engine. 

Since the instrumental phases are largely random, the simplest 
way to remove them is to measure them from a source on the sky. 

\section{FRB VLBI search}

\begin{figure}[H]
\label{fig-bf_diagram}
\begin{center}
\includegraphics[trim={1.in, 1.in, 2.5in, 1.in}, width=1\textwidth]{./figures/beamforming/moose_diagram.png} 
\vspace{0.0cm}
\caption[abc]{}  
\vspace{-0.4cm}   
\end{center}
\end{figure}

\section{Conclusion}
\label{sec:conclusion}
  

% ================================================================================
% ACKNOWLEDGEMENTS
% ================================================================================

\section*{\centering Acknowledgements}

We thank Ondrej and Nolan and Peterman.
  
